% \documentclass{bbe} % Use the custom class

% \begin{document}

% \chapter{Chapter Title Here}
% % Chapter content goes here.

% \section{Section Title Here}
% % Section content goes here.

% \subsection{Subsection Title Here}
% % Subsection content goes here.

% \subsubsection{Subsubsection Title Here}
% % Subsubsection content goes here.

% \begin{activity}
%     % Your activity content here.
% \end{activity}

% \begin{definition}
%     % Your definition content here.
% \end{definition}

% \begin{theorem}
%     % Your theorem content here.
% \end{theorem}

% \begin{example}
%     % Your example content here.
% \end{example}

% \begin{exercise}
%     % Your exercise content here.
% \end{exercise}

% \begin{generality}
%     % Your generality content here.
% \end{generality}

% \begin{property}
%     % Your property content here.
% \end{property}

% \begin{solution}
%     % Your solution content here.
% \end{solution}

% \begin{remark}
%     % Your remark content here.
% \end{remark}

% \begin{note}
%     % Your note content here.
% \end{note}

% % Demonstration of customized lists
% \begin{enumerate}[1] % Default numeric enumeration
%     \item First item.
%     \item Second item.
% \end{enumerate}

% \begin{enumerate}[a] % Alphabetical enumeration
%     \item First item.
%     \item Second item.
% \end{enumerate}

% \begin{itemize}[f1] % Custom itemize with specific symbol
%     \item First item.
%     \item Second item.
% \end{itemize}

% \begin{itemize}[b] % Bullet points
%     \item First bullet point.
%     \item Second bullet point.
% \end{itemize}

% % Example of a multicolumn enumeration
% \begin{Enumerate}(2) % Two-column enumerate
%     \item First item in two-column list.
%     \item Second item in two-column list.
% \end{Enumerate}

% % Custom tasks (similar to itemize or enumerate)
% \begin{tasks}(2) % Two-column tasks
%     \task First task.
%     \task Second task.
% \end{tasks}

% \end{document}


\chapter{Título do Capítulo 1}

\section{Título da Seção}

\begin{definition}
    Olá mundo
\end{definition}

\begin{remark}
    Este é um comentário
\end{remark}

\begin{example}
    Este é um exemplo
\end{example}

\chapter{Título do Capítulo Aqui}
% Conteúdo do capítulo vai aqui.
este é o capítulo
\section{Título da Seção Aqui}
% Conteúdo da seção vai aqui.
esta é a seção
\subsection{Título da Subseção Aqui}
% Conteúdo da subseção vai aqui.
esta é a subseção
\subsubsection{Título da Subsubseção Aqui}
% Conteúdo da subsubseção vai aqui.
esta é a subsubseção
\begin{activity}
Conteúdo da sua atividade aqui.
\end{activity}

\begin{definition}
Conteúdo da sua definição aqui.
\end{definition}

\begin{theorem}
Conteúdo do seu teorema aqui.
\end{theorem}

\begin{example}
Conteúdo do seu exemplo aqui.
\end{example}

\begin{exercise}
Conteúdo do seu exercício aqui.
\end{exercise}

\begin{generality}
Conteúdo da sua generalidade aqui.
\end{generality}

\begin{property}
Conteúdo da sua propriedade aqui.
\end{property}

\begin{solution}
Conteúdo da sua solução aqui.
\end{solution}

\begin{remark}
Conteúdo do seu comentário aqui.
\end{remark}

% Demonstração de listas personalizadas
\begin{enumerate}[1] % Enumeração numérica padrão
    \item Primeiro item.
    \item Segundo item.
\end{enumerate}

\begin{enumerate}[a] % Enumeração alfabética
    \item Primeiro item.
    \item Segundo item.
\end{enumerate}

\begin{itemize}[f1] % Lista com símbolo específico
    \item Primeiro item.
    \item Segundo item.
\end{itemize}

\begin{itemize}[b] % Pontos de bala
    \item Primeiro ponto.
    \item Segundo ponto.
\end{itemize}

% Exemplo de enumeração em várias colunas
\begin{Enumerate}(2) % Lista em duas colunas
    \item Primeiro item na lista de duas colunas.
    \item Segundo item na lista de duas colunas.
\end{Enumerate}

% Tarefas personalizadas (similar a itemize ou enumerate)
\begin{tasks}(2) % Tarefas em duas colunas
    \task Primeira tarefa.
%    \task Segunda tarefa.
%\end{tasks}